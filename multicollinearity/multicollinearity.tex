\documentclass{article}
\usepackage{graphics}
\usepackage{indentfirst}
\usepackage{amsmath}
\usepackage{algorithm}
\usepackage{algorithmic}
\usepackage{bm}
\usepackage{setspace}
\usepackage{graphicx}
\usepackage{float}
\usepackage{CJKutf8}
\usepackage[colorlinks]{hyperref}

\author{Ruichen Wang}
\title{共线性问题}
\begin{document}
\begin{CJK*}{UTF8}{gbsn}

\maketitle

\section{多重共线性 multicollinearity}
\subsection{什么是multicollinearity?}
\href{https://en.wikipedia.org/wiki/Multicollinearity}{wiki词条}
\subsection{对模型拟合有什么坏处?}
通常情况下,现实真实数据之间难免有一些相关性。一点点的数据共线性无伤大雅。但是如果有严重


\section{回归模型}
\subsection{参数选择}
有一种说法是输入模型的特征维度越多越好,人多力量大。但实际上并不是。更多的特征会使得模型学习的目标不清晰,更难以理解。

就像你很难分辨出合唱时谁唱得最好。 

\subsection{如何判断共线性存在?}


\section{评价指标与解决}

\end{CJK*}
\end{document}

























