\documentclass{article}
\usepackage{graphics}
\usepackage{indentfirst}
\usepackage{amsmath}
\usepackage{algorithm}
\usepackage{algorithmic}
\usepackage{bm}
\usepackage{setspace}
\usepackage{graphicx}
\usepackage{float}
\usepackage{CJKutf8}
\usepackage{hyperref}

\author{Ruichen Wang}
\title{FM}
\begin{document}
\begin{CJK*}{UTF8}{gbsn}

\maketitle
\begin{abstract}
Factorization machines 因子分解机基础介绍,以及相关模型与思考。
\end{abstract}

\tableofcontents

\section{Factorization Machines (FM)}
目前常见的工业推荐系统会分为召回排序两个阶段,是因为这两个阶段各司其职,职责分明。召回主要考虑泛化性并把候选物品集合数量降下来;排序则主要负责根据用户特征、物品特征、上下文特征对物品进行精准排名。

\begin{figure}[H]
\centering
\includegraphics[width=4.8in,height=2.4in]{recsys}
\caption{推荐系统的两个模块}
\end{figure}

在介绍具体内容之前,\textbf{一起思考下面几个问题}:
\begin{enumerate}
\item{多路召回有什么优势?有什么缺点?}
\item{单路召回行不行?能不能用一个统一的模型来将多路召回改成单路召回?}
\item{能不能将召回阶段与排序阶段整合起来?有什么困难、不同?}
\item{多路召回如何选择K值?能否端到端优化?是否需要用户分层?}
\item{不使用FM or DeepFM,直接使用DNN行不行?}
\end{enumerate}


FM \cite{DBLP:conf/icdm/Rendle10} 主要被用来处理高稀疏的特征。有线性的计算复杂度。 在实际应用中常用来做排序。

\paragraph{FM模型} 假设 $x \in R^{n}$, 待计算的模型参数 $ w_{0} \in R, \textbf{w} \in R^{n}, \textbf{V} \in R^{n \times k}$.
$$\widehat{y}= w_{0}+\sum_{i=1}^{n}w_{i}x_{i}+\sum_{i=1}^{n}\sum_{j=i+1}^{n}\langle \textbf{v}_{i} ,\textbf{v}_{j} \rangle x_{i}x_{j}$$
where 
$$\langle \textbf{v}_{i} ,\textbf{v}_{j} \rangle =\sum_{m=1}^{k}v_{i,m} \cdot v_{j,m}$$

\paragraph{数学变换} 原问题的复杂度是$O(kn^{2})$. 通过一些数学变换可以优化到线性的复杂度 $O(kn)$. 达到了和LR接近的性能。
\begin{align*}
\sum_{i=1}^{n}\sum_{j=i+1}^{n}\langle \textbf{v}_{i} ,\textbf{v}_{j} \rangle x_{i}x_{j}
&= \frac{1}{2} \sum_{i=1}^{n}\sum_{j=1}^{n}\langle \textbf{v}_{i} ,\textbf{v}_{j} \rangle x_{i}x_{j} -\frac{1}{2}\sum_{i=1}^{n}\langle \textbf{v}_{i} ,\textbf{v}_{i} \rangle x_{i}x_{i} \\
&= \frac{1}{2}\sum_{m=1}^{k}\left( \left(\sum_{i=1}^{n}v_{i,m}x_{i}\right)^{2} -\sum_{i=1}^{n}v_{i,m}^{2}x_{i}^{2} \right)
\end{align*}

*哪来的$\frac{1}{2}$?

\section{LR-SVM-FM}
LR的特点: 模型简单,容易解释,规模弹性,人工构造or组合特征,学习一阶特征权重.
$$\widehat{y}=\sigma(w^{T}x)$$

线性核linear kernel SVM: $K(x,z)=1+<x,z>$。这里等价于d=1的FM
$$\phi(x)=(1,x_{1},...,x_{n})$$
$$\widehat{y}=w_{0}+\sum_{i}^{n}w_{i}x_{i}$$

多项式核 polynomial kernel: $K(x,z)=(1+<x,z>)^{d}$, d=2时
$$\phi(x)=(1,\sqrt{2}x_{1},...,\sqrt{2}x_{n},x_{1}^2,...,x_{n}^2,\sqrt{2}x_{1}x_{2},...,\sqrt{2}x_{n-1}x_{n})$$
多项式SVM可以写成:
$$\widehat{y}= w_{0}+\sqrt{2}\sum_{i=1}^{n}w_{i}x_{i}+\sum_{i=1}^{n}w_{i}^{2}x_{i}^{2}+\sqrt{2} \sum_{i=1}^{n}\sum_{j=i+1}^{n} w_{i,j}^{2} x_{i}x_{j}$$
\noindent
\textbf{多项式SVM同样是二阶特征,为什么这样不好?}

要学习到一个足够可靠的$w_{i,j}$, 需要足够的$(i,j)$ case。 只要用户i 或者商品j 有一个为0, 就没有办法学习$w_{i,j}$。 如果数据非常稀疏,那么就意味着没有足够的case来学习$w_{i,j}$。

\begin{itemize}
\item SVM需要数据相对稠密,$(i,j)$之间交互要足够多。
\item SVM学习常需要转化成对偶形式,FM可以直接求解
\end{itemize}

\section{Matrix Factorization(MF)}
Matrix Factorization 矩阵分解的核心思想是通过两个低维小矩阵的乘积计算,来模拟真实用户点击或评分产生的大的协同信息稀疏矩阵,本质上是编码了用户和物品协同信息的降维模型。

\noindent
\textbf{和FM有什么不一样?}

MF可以被认为是只有User ID 和Item ID这两个特征的FM模型,MF将这两类特征通过矩阵分解,学习到user 和item 的embeddings. 而FM可以看作是扩展的MF, 可以加入更多的side info. 

如果模型能引入其他信息(只考虑ID),明显考虑受限,很不实用。这也是为何矩阵分解类的方法很少看到在Ranking阶段使用,通常是作为一路召回形式存在的原因。
\section{Field-aware Factorization Machines (FFM)}
FFM \cite{DBLP:conf/recsys/JuanZCL16} 主要是针对不同特征,细分到不同的field再学习一个独立的embedding.
$$\phi_{FFM}(\textbf{w},\textbf{x})=\sum_{i=1}^{n} \sum_{j=i+1}^{n}(\textbf{w}_{i,f_{j}},\textbf{w}_{j,f_{i}})\textbf{x}_{i} \textbf{x}_{j}$$

\begin{figure}[H]
\centering
\includegraphics[width=4in,height=0.6in]{ffm1}
\caption{FFM 例子}
\end{figure}
对于之前的FM来说:
$$\phi_{FM}(w,x)=w_{ESPN}\cdot w_{Nike}+w_{ESPN}\cdot w_{Male}+w_{Nike}\cdot w_{Male}$$
对于不同的field (广告 or 性别),$w_{ESPN}$其实用的都是同一个向量。

FFM这些向量做了更细致的特征表达,对于每一个不同的field,都学习一个向量(Field-aware)。
$$\phi_{FFM}(w,x)$$

\section{DeepFM}
\section{Answers ?}
\begin{enumerate}

\item \textbf{多路召回有什么优势?有什么缺点?}

\textbf{多路召回的缺点}:
\begin{itemize}
\item 不同策略召回的item打分不能统一比较,所以需要靠ranking模型来进行打分。
\item 多路召回的另一个问题就是,每一路应该选多少候选集?也就是K值的选择。每一路的K值其实是超参数,线上需要不断调整,其实我们并不知道最优的k的组合是什么。理解的情况,每个用户对不同路的召回兴趣是不同的,所以不同路的召回应该有不同的K值。
\item 在排序ranking部分也有可能产生一些问题,新增的召回策略有可能没有把相对应的特征加到ranking中,导致新增召回路看上去没什么用,因为即使你找回来了,而且用户真的可能点击,但是在排序阶段死活排不上去。
\end{itemize}

\textbf{多路召回的优点}:
\begin{itemize}
\item 上线新召回算法比较灵活
\item 不同路召回之间没有耦合关系,上线一个召回不会影响其他模型
\end{itemize}

\item \textbf{单路召回行不行?能不能用一个统一的模型来将多路召回改成单路召回?}

可以。 FM构造一个单路召回的模型
$$\widehat{y}=FM(User,Item,Context)$$

\begin{figure}[H]
\centering
\includegraphics[width=4.8in,height=2in]{fm1}
\caption{最简单的FM模型,暂不考虑context}
\end{figure}
结合context: 一种思路可以是计算context特征的sum average,然后通过$U+C$去召回出Item,根据$<U,C>$进行Item排序。
\item \textbf{能不能将召回阶段与排序阶段整合起来?有什么困难、不同?}

要将这两块整合,主要考虑两个方面:1、速度(海量数据的查询,例如 similiarity search topk embeddings的速度)。2、精度(没有了粗排之后,精度还能否有保障?)

如果是在排序阶段使用FM/FFM或者其他模型,因为此时用户已知,要排序的具体是哪篇文章也知道,都是少量数据,此时模型的任务是要判断用户是否对某篇文章感兴趣,所以用户特征和物料特征可以同时作为模型的输入。

而如果是在召回阶段使用FM/FFM模型,user的信息是有的,但是item的信息往往是千万量级的,要求模型在海量数据中找到那一小批用户感兴趣的item出来,而且要保证速度。如何计算才能满足FM/FFM的思想呢?


\end{enumerate}

\bibliographystyle{plain}
\bibliography{ref}

\end{CJK*}
\end{document}

























